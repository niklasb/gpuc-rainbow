\documentclass[ngerman,hyperref={pdfpagelabels=true}]{beamer}
\usepackage{etex}

% -----------------------------------------------------------------------------

\graphicspath{{img/}}

% -----------------------------------------------------------------------------

\usetheme{KIT}

\setbeamercovered{transparent}
\setbeamertemplate{enumerate items}[ball]

\usepackage[ngerman]{babel}
\usepackage[utf8]{inputenc}
\usepackage[TS1,T1]{fontenc}
\usepackage{array}
\usepackage[absolute,overlay]{textpos}
\usepackage{beamerKITdefs}
\usepackage{graphicx}
\usepackage{tikz}
\usepackage{listings}
\usepackage[loadonly]{enumitem}
\usepackage{mathtools}

\usetikzlibrary{trees,shapes}

\pdfpageattr {/Group << /S /Transparency /I true /CS /DeviceRGB>>}
\pdfpageattr {/Group << /S /Transparency /I true /CS /DeviceRGB>>}


\title{Rainbow Tables}
\subtitle{Abschlussproject ,,GPU Computing'' -- Niklas Baumstark}


\author{Niklas Baumstark}
\institute{Lehrstuhl für Computergrafik}

\TitleImage[width=\titleimagewd,height=\titleimageht]{titel.jpg}

\KITinstitute{ITI}
\KITfaculty{Fakultät für Informatik}

\lstset{
basicstyle=\ttfamily\small
}

% ----------------------------------------------------------------------
\begin{document}
% ----------------------------------------------------------------------

%\AtBeginSection[]
%{\frame{\frametitle{Outline}
%\tableofcontents[currentsection]}}

\setlength\textheight{7cm} %required for correct vertical alignment, if [t] is not used as documentclass parameter

% title frame
\begin{frame}
  \maketitle
\end{frame}

\newcommand\BigO{\mathcal{O}}

% title frame
\begin{frame}{Sichere Speicherung von Passw"ortern}
  \begin{itemize}
  \item Viele User verwenden selbes Passwort f"ur verschiedene Dienste
  \item Szenario: Kompromittierter Server im Internet
  \begin{itemize}
  \item Angreifer kann Datenbank auslesen
  \item Angreifer kann u.U. live eingegebene Passw"orter mitschneiden
  \end{itemize}
  \item Ziel: Angreifer kann inaktive Passw"orter nicht lesen
  \item L"osung: Vor Speicherung Hashfunktion $h$ auf Passwort anwenden
  \end{itemize}
\end{frame}

\begin{frame}{Als Angreifer}
  \begin{block}{Gegeben}
  Passwortraum $P$, Hashraum $H$, Hashfunktion $h: P \rightarrow H$, Hashwert $y \in H$
  \end{block}
  \begin{block}{Gesucht}
  Urbild $p \in P$ mit $h(p) = y$
  \end{block}
  \begin{block}{Kein Speicher $\rightarrow$ $\BigO(|P|)$ Zeit pro Anfrage}
  Iteriere "uber $P$ bis wir Urbild finden
  \end{block}
  \begin{block}{$\BigO(|P|)$ Speicher $\rightarrow$ $\BigO(1)$ Zeit pro Anfrage}
  Baue einmalig Datenbank aller Hashwerte auf
  \end{block}
\end{frame}

\newlength{\arrow}
\settowidth{\arrow}{\scriptsize$10000$}
\newcommand*{\ra}[1]{\xrightarrow{\mathmakebox[\arrow]{#1}}}

\begin{frame}{Rainbow Tables}
  \begin{itemize}
  \item Rainbow Tables bieten Tradeoff zwischen Speicher und Abfragezeit~\cite{oechslin03}
  \item Benutze Familie von Reduktionsfunktionen $r_i: H \rightarrow P$
  \item Berechne Ketten von Werten die zwischen $H$ und $P$ alternieren, ausgehend
  von Startwerten $x_0 \in P$

\[
x_0 \ra{h} h_0 \ra{r_0} x_1 \ra{h} h_1 \ra{r_1} x_2 \ra{h} \ldots \ra{r_{t-1}} x_t
\]

  \item W"ahle verschiedene $x_0$, speichere f"ur jede Kette nur $x_0$ und $x_t$
  \item Implizit wird auch die Menge aller $x_j$ ($1 \le x_j < t$) repr"asentiert
  \item Bei Hashabfrage partielle Ketten aufbauen und Endpunkte in der Datenbank
  nachschlagen
  \end{itemize}
\end{frame}

\begin{frame}{Kollisionen und Merges}
  \begin{itemize}
  \item Wenn $x_i$ f"ur festes $i$ in verschiedene Ketten "ubereinstimmt kollidieren
  sie ab diesem Punkt bis zum Ende $\rightarrow$ \emph{Merge}
  \item Ziel: Hohe Abdeckung von $P$, also m"oglichst wenige Merges
  \item \emph{Perfekte} Tabelle hat gar keine Merges, alle Endpunkte verschieden
  \item Ketten k"onnen auch f"ur $x_i$, $x_j$ mit $i \neq j$ kollidieren, wirkt
  sich aber weniger schlimm auf Abdeckung aus
  \end{itemize}
\end{frame}

\begin{frame}{Das Projekt}
  \begin{itemize}
  \item Aufbauen und Abfragen von perfekten Rainbow Tables mithilfe der GPU,
  "uber Alphabet $\Sigma$
  \item Ziele
  \begin{itemize}
    \item Speichereffizienter Aufbau der Tabelle im Device Memory
    \item Entfernen von Duplikaten w"ahrend des Aufbaus
    \item Kompetitiver Durchsatz
  \end{itemize}
  \end{itemize}
\end{frame}

\begin{frame}{Berechnung der Ketten}
  \begin{itemize}
  \item Stelle Strings als Zahlen zur Basis $|\Sigma|$ dar $\rightarrow$ Basisumwandlung
  \item Ein Work Item pro Kette
  \item Mikrooptimierungen essentiell, besonders Vermeiden von Branch Divergence
  \item Duplikatentfernung
  \begin{itemize}
    \item F"uhre alle $X$ Iterationen durch
    \item Schritt 1: Sortiere Ketten nach Endpunkt (Bitonic Sort)
    \item Schritt 2: Markiere Ketten deren Endpunkt gleich dem des Vorg"angers ist
    \item Schritt 3: Filtere alle markierten Ketten aus mit Prefix Scan~\cite{blelloch10}
    \item Nebeneffekt: Tabelle am Ende schon sortiert
  \end{itemize}
  \end{itemize}
\end{frame}

\begin{frame}{Hashabfrage}
  \begin{itemize}
  \item Berechne partielle Ketten f"ur alle Queries, nach L"ange sortiert
  \item Sortiere Endpunkte der partiellen Ketten
  \item Benutze Bin"arsuche um Startpunkte zu finden
  \item Rekonstruiere Ketten um Hashwert zu finden
  \end{itemize}
\end{frame}

\begin{frame}{Evaluation}

F"ur $t = 1000, |\Sigma| = 36, n = 6$
\\[1em]

  \begin{tabular}[]{l | r | r}
Programm        & Hashreduktionen pro Sek. & Abfragen pro Sek. \\ \hline
RainbowCrack    & 5.0M   &  16.6 \\
niklas-cpu      & 4.6M   &  12.0 \\
niklas-ocl      & 662M   &  714.0 \\
CryptoHaze-ocl  & 735M   &  (51.0) \\
CryptoHaze-cuda & 1070M  &  (100.0)
  \end{tabular}

\end{frame}

\begin{frame}{Referenzen}
\bibliographystyle{apalike}
\bibliography{references}
\end{frame}

\end{document}
